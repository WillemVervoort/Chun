\documentclass[]{elsarticle} %review=doublespace preprint=single 5p=2 column
%%% Begin My package additions %%%%%%%%%%%%%%%%%%%
\usepackage[hyphens]{url}

  \journal{Journal of Hydrology Regional Studies} % Sets Journal name


\usepackage{lineno} % add
\providecommand{\tightlist}{%
  \setlength{\itemsep}{0pt}\setlength{\parskip}{0pt}}

\bibliographystyle{elsarticle-harv}
\biboptions{sort&compress} % For natbib
\usepackage{graphicx}
\usepackage{booktabs} % book-quality tables
%%%%%%%%%%%%%%%% end my additions to header

\usepackage[T1]{fontenc}
\usepackage{lmodern}
\usepackage{amssymb,amsmath}
\usepackage{ifxetex,ifluatex}
\usepackage{fixltx2e} % provides \textsubscript
% use upquote if available, for straight quotes in verbatim environments
\IfFileExists{upquote.sty}{\usepackage{upquote}}{}
\ifnum 0\ifxetex 1\fi\ifluatex 1\fi=0 % if pdftex
  \usepackage[utf8]{inputenc}
\else % if luatex or xelatex
  \usepackage{fontspec}
  \ifxetex
    \usepackage{xltxtra,xunicode}
  \fi
  \defaultfontfeatures{Mapping=tex-text,Scale=MatchLowercase}
  \newcommand{\euro}{€}
\fi
% use microtype if available
\IfFileExists{microtype.sty}{\usepackage{microtype}}{}
\usepackage{longtable}
\ifxetex
  \usepackage[setpagesize=false, % page size defined by xetex
              unicode=false, % unicode breaks when used with xetex
              xetex]{hyperref}
\else
  \usepackage[unicode=true]{hyperref}
\fi
\hypersetup{breaklinks=true,
            bookmarks=true,
            pdfauthor={},
            pdftitle={Detecting influences of changes in land cover on observed rainfall.},
            colorlinks=true,
            urlcolor=blue,
            linkcolor=magenta,
            pdfborder={0 0 0}}
\urlstyle{same}  % don't use monospace font for urls

\setcounter{secnumdepth}{5}
% Pandoc toggle for numbering sections (defaults to be off)
% Pandoc header



\usepackage{amsthm}
\newtheorem{theorem}{Theorem}[section]
\newtheorem{lemma}{Lemma}[section]
\theoremstyle{definition}
\newtheorem{definition}{Definition}[section]
\newtheorem{corollary}{Corollary}[section]
\newtheorem{proposition}{Proposition}[section]
\theoremstyle{definition}
\newtheorem{example}{Example}[section]
\theoremstyle{definition}
\newtheorem{exercise}{Exercise}[section]
\theoremstyle{remark}
\newtheorem*{remark}{Remark}
\newtheorem*{solution}{Solution}
\begin{document}
\begin{frontmatter}

  \title{Detecting influences of changes in land cover on observed rainfall.}
    \author[a]{Chun X. Liang}
   \ead{summer.chunliang@gmail.com} 
  
    \author[a]{Floris F. van Ogtrop}
   \ead{floris.vanogtrop@sydney.edu.au} 
  
    \author[a]{R. Willem Vervoort\corref{c1}}
   \ead{willem.vervoort@sydney.edu.au} 
   \cortext[c1]{Corresponding Author}
      \address[a]{Sydney Institute of Agriculture, The University of Sydney, NSW 2006}
  
  \begin{abstract}
  Analysis of observational data to identify relationships between
  rainfall and land cover change are difficult due to multiple
  environmental factors that cannot be strictly controlled. In this study
  we present a methodology to investigate the relationship using
  statistical methods on data from best available sources at two sites in
  Australia. Gridded data of rainfall and tree cover were used as
  spatially corresponding local conditions. Large scale effects were
  represented by climatic indicators, such as SOI and IOD. Regression
  analysis and step trend tests were used to assess the effect of abrupt
  land surface intervention. At a Queensland site, significant tree cover
  change between 2002 - 2005 did not result in strong statistically
  significant precipitation changes. On the other hand, results from a
  bushfire affected NSW/VIC region suggests significant changes in the
  rainfall. This indicates the method works better when a abrupt change in
  the data can be clearly identified. The results from the step trend test
  implied a positive relationship between the tree cover and the rainfall
  at 0.1 significance level in both locations in data up to 2009. However,
  high rainfall variability and possible regrowth meant that no signifcant
  changes were observed in alonger time sereies to 2015.
  \end{abstract}
  
 \end{frontmatter}

\section{Introduction}\label{introduction}

Land use and land cover changes can lead to changes in the local
climate. Empirical and modelling studies have found cloud types and
rainfall are correlated to large scale vegetation cover changes, such as
deforestation in the Amazon and in the Sahel (Chagnon and Bras 2005;
Pinto et al. 2009; Wang et al. 2009; Mei and Wang 2010; Kucharski, Zeng,
and Kalnay 2013; Pitman and Lorenz 2016) and afforestation in south
Israel (Otterman et al. 1990; Ben-Gai et al. 1998). Using airborne
measurement in Western Australia, Junkermann et al. (2009) showed a
significantly higher level of aerosols over an agricultural area
compared to an adjacent natural vegetation. They suggested that a
modification of aerosol concentrations due to deforestation could have
contributed to a reduction of local rainfall, as more, but smaller rain
droplets were observed. Nair et al. (2011) reported from the Bunny Fence
Experiment in Western Australia that local land use change altered the
synoptic west coast trough dynamics and surface roughness, and this
resulted in an observed rainfall decrease. Maximum temperatures were
also found to be sensitive to land cover change in eastern Australia
(McAlpine et al. 2007).

Overall the number of empirical studies analyzing changes to rainfall
due to land cover change from observational data is limited. Most of the
studies mentioned previously were either model simulations, or
comparisons of modelled data with observations. This is because there
are some fundamental experimental difficulties in both space (where does
evaporated water reappear as rainfall?) and in time (how much time does
it take for land cover change effects to appear or disappear?). In
addition, in many areas across the globe rainfall variability is related
to complex set of interactions, of which land use change might only be a
minor component.

Locally, there are two main sources that generate rainfall: moisture
from advective atmospheric transport; and local evapotranspiration
(Eltahir and Bras 1996; Bosilovich and Chern 2006; Dirmeyer, Brubaker,
and DelSole 2009; Gimeno et al. 2010). The local evapotranspiration
component is the component considered to be affected by land use change
(Eltahir and Bras 1996). According to Trenberth (1999), the contribution
of advective moisture partially depends on the availability of external
moisture and atmospheric transport. On the longer time scale, such as
monthly and annually, large scale atmospheric dynamics are affected by
large scale climate drivers. For example, many studies have reported
significant relationships between rainfall in large parts of Australia
and the El Niño-Southern Oscillation (ENSO) (Verdon et al. 2004; Risbey
et al. 2009; Speer, Leslie, and Fierro 2011). In contrast, local ET is
determined by local land surface characteristics, which influence local
scale atmospheric dynamics and hence the amount of rainfall, including
contribution from both main sources. Therefore land surface is suggested
to play an important role in local rainfall.

Although climate drivers demonstrate some capability to predict
Australian rainfall, there is still a large amount of unexplained
variance. Westra and Sharma (2010) pointed out that models based on
global sea surface temperature anomalies can only predict up to 14.7\%
of annual precipitation variance. Some of the remaining variance could
be due to land surface processes as suggested in studies predicting
local rainfall (e.g. Ma et al. 2011; Zeng et al. 2012; Pitman and Lorenz
2016; Saha, Dirmeyer, and Chase 2016). However, they are mostly based on
modelling experiments and little evidence was reported from
observations. However, Pitman et al. (2004) found a good match between
observations and simulated rainfall changes in southwest Western
Australia ,forced by land cover change. Timbal and Arblaster (2006) were
able to reproduce the rainfall decline in south west Australia by
including land cover influence. In addition, local land use change might
not be a primary, but is likely to be a secondary cause of rainfall
change (Nicholls 2006).

Therefore, the aim of this study is to expand the available methodology
to use empirical evidence at regional scales to investigate the cause
and effect relationship between land cover change and local rainfall.
More specifically, we hypothesize that a step change on the land cover
on the surface will cause a step change in the rainfall. To demonstrate
the approach and this effect we applied the methodology to study the
changes in rainfall at two locations in Queensland and NSW/Victoria
where there are possible step changes in land cover change due to land
clearing and bush fires. The methodology is mainly based on statistical
approaches to identify changes in rainfall, which were subsequently
associated with land cover change through spatial comparison.

In this paper, after this section (the introduction), section 2 covers
the case study areas and the observed land use change. Section 3
describes the data used in the study in more detail. Section 4 details
the statistical methods and the underlying assumptions related to the
modelling approach, Section 5 gives the results, which are further
discussed in section 6 and finally section 7 offers the conclusions.

\section{Study regions and tree cover
change}\label{study-regions-and-tree-cover-change}

We use two areas in Australia to demonstrate the approach. In Australia,
significant tree cover change has mainly occurred in the north east of
the continent and on the southeast coast, as well as in the southwest of
Western Australia. According to the National Dynamic Land Cover Dataset
(DLCD) (Lymburner et al. 2010), most of these areas have experienced
decreasing EVI post 2000. As an index for vegetation greenness, the
decreasing values indicate lower biomass over time in the tree cover
regions. The possible EVI reduction might be due to land clearing, bush
fires or drought.

\begin{figure}
\includegraphics[width=0.9\linewidth]{figures/map_selreg} \caption{Selected study regions are highlighted by red rectangles in the main map (the red rectangle in the insert indicates the location of the main map). The types of tree cover in 2008 from the DLCD product is shown at the background. In site 1 (the QLD region), the tree cover is mostly sparse. In site 2 (the NSW/VIC region), many areas have open or close forest in which tree cover is denser.}\label{fig:selreg}
\end{figure}

Two regions were selected where significant tree cover change was
reported. The first region is located in south central Queensland to the
north of the Murray Darling Basin (MDB) (site 1 in Figure
\ref{fig:selreg}. High rates of land clearing have been reported in this
region during the early 2000s (Department of Natural Resources and Water
2007). The second study region is located at the border of New South
Wales and Victoria, and includes the Snowy Mountain ranges (site 2 in
Figure \ref{fig:selreg}. Severe bush fires occurred in this area and the
surroundings in early 2003 (see Figure \ref{fig:figure2bushfire}). The
2003 bush fires were the largest and the worst in this area for the last
60 years (The State Government of Victoria 2011). Two thirds of
Kosciuszko national park was heavily burned and regrowth was reported to
be slow due to drought and cold conditions (ABC News 2003) and the type
of species in this region. However, in the longer term, after an early
high transpiration period a recovery of pre-fire evapotranspiration
would be expected (Kuczera 1987). As a result , significant tree cover
loss has happened in both study areas in the last decade, either
permanently or temporarily.

\begin{figure}
\includegraphics[width=0.9\linewidth]{figures/bushfire_nswvic} \caption{Location of bushfires occurring in January 2003, in and around the NSW/VIC study region, as shown by the red pixels. The map shows large area in the Kosciuszko national park has been burned. Some locations in the southwest of ACT have also experienced intensive bushfires.}\label{fig:figure2bushfire}
\end{figure}

The two regions have different climate characteristics. The QLD region
is partially grassland and partially subtropical, while the NSW/VIC
region is mainly within the temperate zone, under the Köppen
classification. According to Australian Bureau of Meteorology (BoM), the
NSW/VIC region receives 1000 - 2000 mm rainfall annually, which is more
than double of the rainfall in the QLD region. Evapotranspiration is
similar in both regions. Marine moisture and orographic effects are
likely to be the main contributors to rainfall in the southeast mountain
areas of the NSW/VIC region.

The land use and land cover characteristics in the two regions are also
different. In the Queensland region, the tree cover is sparse over most
of the area. The MODIS satellite tree cover data (discussed in more
detail in section 3) shows that tree cover in this region is generally
below 20\% of total ground area. Grazing is the main activity in this
region, with over 90\% of land used by the grazing industry (ABARES
2010). Our starting assumption is that the main cause of the EVI decline
over large part of the region is due to land clearing. Tree cover has
been cleared at a massive scale over the last decade, especially during
2002 - 2004. The Kosciuszko national park is within the NSW/VIC region.
Here tree cover is denser with open or even closed forest (the tree
cover distribution is bimodal at 10 - 20\% and 60 - 70\%). The dominant
species in the alpine area are Snow Gum and large stand species such as
Alpine Ash and Mountain Gum in the sub-alpine area. These trees can
reach a great height but they take long time to grow. For example,
Alpine Ash would need about 20 years to mature. Although land clearing
is not a major issue in this region, it is vulnerable to fires and
drought.

Therefore two types of land cover changes were studied. The reports from
the Queensland Statewide Land Cover and Trees Study (SLATS) (e.g.
Department of Natural Resources and Mines 2005; Department of Science,
Information Technology and Innovation 2017) were used to investigate the
time and location of land clearing in the QLD region. The MODIS burned
area product, MCD45A1 (Roy, Lewis, and Justice 2002; Roy et al. 2005;
Roy et al. 2008), was used to locate bush fires areas in the NSW/VIC
region, with a grid resolution of 500 m. MCD45A1 provides monthly
burning information on all pixels, which helps to pinpoint an abrupt
event. Due to the nature of the different land cover change, the
post-change vegetation status in the two regions is expected to be
different (see Figure \ref{fig:figure3-tc-simple}).

\begin{figure}
\includegraphics[width=0.9\linewidth]{figures/tc_simple} \caption{The expected evolution of the land surface after trees have been removed in (a) the QLD region and (b) the NSW/VIC region.}\label{fig:figure3-tc-simple}
\end{figure}

The hypothesis in this study therefore is that the effect of 2003 - 2004
land clearings in the QLD region and the 2003 bush fires in the NSW/VIC
region cause a step change in the local rainfall. The actual tree cover
change at the pixel level during this time was derived from the 15-year
MODIS data (discussed below). The difference of tree cover before and
after the land disturbance was tested using a Student's t-test. As the
length of the tree cover data is shorter than available rainfall data,
earlier land clearings in the QLD region cannot be identified spatially,
hence they are excluded from the analysis.

\hypertarget{Data}{\subsection{Data}\label{Data}}

Several land surface data sets were used in this study. The main one was
the MOD44B product Global Vegetation Continuous Field data set (version
5). This data set provides estimates of percent tree cover (percentage
of ground surface covered by trees) at a grid resolution of 250 m
(Townshend et al. 2011), which is finer then the earlier mentioned
burned product MCD45A1. The data set is available on an annual time
interval for the study period of 2000 - 2015. The tree cover data was
produced from 16-day Terra MODIS Land Surface Reflectance data and Land
Surface Temperature (Townshend et al. 2011). The National Dynamic Land
Cover Dataset (DLCD) (Lymburner et al. 2010) from the Australian
Collaborative Land Use Mapping Program (ACLUMP) was used to verify the
trend of vegetation cover change calculated from the previous data set.
This data set, developed by Geoscience Australia and Australian Bureau
of Agricultural and Resource Economics and Sciences (ABARES), is the
first nationally consistent and thematically comprehensive land cover
reference for Australia. The DLCD is based on the 16-day Enhanced
Vegetation Index (EVI), again from the MODIS satellite, between April
2000 and April 2015. It also has a grid resolution of 250 m. The data
set provides information on the final land cover types (as in 2015) and
estimated trend of EVI statistics (annual mean, maximum and minimum).

Rainfall data for Australia (Jones, Wang, and Fawcett 2009) was obtained
from the Bureau of Meteorology. The data has been projected onto a
national 0.05\textdegree × 0.05\textdegree grid (approximately 5 km × 5
km). This gridded data set was generated from station observations using
an optimised Barnes successive correction technique. The Barnes
technique combines a weighted averaging process and defined
topographical information to estimate rainfall values between spatial
points (BoM 2009). The resulting data set provides additional
information for data-sparse areas like central Australia but reduces
information in the data-rich areas, such as southeast Australia where
station density is up to 20 per 100 \(\text{km}^2\). The data is
available on a monthly basis from 1900 to current. Here a subset of 36
years (1979 - 2015) was used. The study was conducted on monthly data,
as a land cover change effect on annual rainfall might be negligible but
can often found to be significant in particular months or seasons (e.g.
Otterman et al. 1990; Gaertner et al. 2001; Semazzi and Song 2001;
Oleson et al. 2004; Deo et al. 2009).

Large scale climate drivers are represented by various climatic indices.
The Southern Oscillation Index (SOI) is generally regarded as a good
predictor of Australian rainfall (Risbey et al. 2009; Chowdhury and
Beecham 2010; Westra and Sharma 2010), but its skill is weaker in some
parts of Australia. For example the Southern Annular Mode (SAM) is found
to be more important than ENSO in south Western Australia (Meneghini,
Simmonds, and Smith 2007). The suitability of each index for the regions
of interest was tested in section @ref(sec:reg\_model). The following
climate indices were used as candidate predictors for local rainfall.

\begin{itemize}
\tightlist
\item
  Southern Oscillation Index (SOI). The Troup version of the monthly SOI
  series used in this study was obtained from BoM (available online at
  \url{http://www.bom.gov.au/climate/current/soihtm1.shtml}).\\
\item
  Eastern, East Central and Central Tropical Pacific Sea Surface
  Temperatures (NINO 3, NINO 3.4 and NINO 4). Monthly SST anomalies are
  available from IRI/LDEO data library and the extended NINO data set is
  used (available online at
  \url{http://iridl.ldeo.columbia.edu/SOURCES/.Indices/.nino/.EXTENDED/}).\\
\item
  Pacific Decadal Oscillation (PDO). The Pacific Decadal Oscillation is
  the leading principal component of monthly SST anomaly in the North
  Pacific Ocean.. The monthly PDO series was provided by JISAO (Joint
  Institute for the Study of the Atmosphere and Ocean, University of
  Washington) (available online at
  \url{http://jisao.washington.edu/pdo/PDO.latest}).\\
\item
  Indian Ocean Dipole (IOD). The Indian Ocean dipole is commonly
  measured by the difference between SST anomaly in the western (50 -
  70\textdegree E and 10\textdegree S-10\textdegree N) and eastern (90 -
  110\textdegree E and 0 - 10\textdegree S) equatorial India Ocean (Saji
  et al. 1999). Monthly IOD was obtained from JAMSTEC (the Japan Agency
  for Marine-Earth Science and Technology) (available online at
  \url{http://www.jamstec.go.jp/frcgc/research/d1/iod/DATA/dmi.monthly.txt}).
\end{itemize}

\subsection{Statistical method}\label{statistical-method}

The step changes on rainfall were analysed using two different
statistical methods. A step change is not obvious in the time series
data of the rainfall residuals (Figure \ref{fig:ts-mean}), even though
the data is deseasonalised and detrended. Both methods in this paper
make use of a regression model to remove variability in rainfall due to
climatic factors to strengthen the tree cover change signal. In the
first method, the tree cover change was implemented as a factor variable
in the regression model. In the second method, a rank sum test (step
trend test), was applied to the model residuals after effects of other
major factors were removed. This assumes that after removal of all
climate variation, the vegetation cover change is the only factor
explaining the non-random pattern in the rainfall residuals.

\begin{figure}
\includegraphics[width=0.9\linewidth]{figures/Rainfall_resid} \caption{The deseasonalised and detrended rainfall over the 30 years period in (a) the QLD region and (b) the NSW/VIC region. The vertical red lines indicate the year of 2003, in which the studied land cover changes occurred. A change in the time series data is not obvious before and after the land cover changes.}\label{fig:ts-mean}
\end{figure}

\subsubsection{Regression model}\label{reg_model}

As highlighted in the introduction, the Australian climate is influenced
by sea surface temperatures in the tropical Pacific and Indian Oceans,
as well as pressure systems in the Southern Ocean (BoM 2012b). Risbey et
al. (2009) compared five large-scale drivers, including ENSO (measured
by SOI and the Tropical Pacific SSTs), IOD, SAM, MJO (Madden-Julian
oscillation) and blocking, in relation to Australian rainfall
variability. The MJO is a large scale eastward-propagating wave-like
disturbance in equatorial latitudes (Risbey et al. 2009). They
identified SOI as the most important index among all indices tested for
broad parts of Australia (including QLD and NSW/VIC) in almost any
season. In this study, four indices were selected from the main climatic
indicators (see section \protect\hyperlink{Data}{Data}) were used as the
explanatory variables in the model for each study region.

Correlations between rainfall and each climate index were analysed.
Rainfall in each study region was first deseasonalised and detrended
using the seasonal decomposition function \texttt{ds} in the package
\texttt{deseasonalise} in R (R Core Team, 2018). Using detrended data
gives a better indication of the underlying correlation rather than the
correlation between trends in the data (Smith and Timbal 2012). The
cross-correlations between the deseasonalised and detrended rainfall and
the climatic indices were tested using the Pearson's product moment
correlation method, assuming the relationships are linear. Although the
optimal technique for exploring the correlation with each index could be
different as described in Risbey et al. (2009), the Pearson's method was
applied to all indices for consistency. Because the PDO describes the
multi-decadal SST with lower frequency (MacDonald and Case 2005;
Zanchettin et al. 2008; Kamruzzaman, Beecham, and Metcalfe 2011),
instead of 37-year rainfall data, a longer period (115 years, from 1900
to 2015) was used to estimate the correlation with PDO, up to lag 24.
For the other indices, the 37-year data was used.

\begin{figure}
\includegraphics[width=0.9\linewidth]{figures/cor_qld} \caption{Cross-correlation of six climate indices and rainfall in QLD study region. For the PDO analysis, 108-year rainfall data (1900 - 2008) are used. Otherwise, 36-year rainfall data are used. The correlation with NINO 3 is not shown as it is very similar to but weaker than for NINO 3.4.}\label{fig:cor-rain-qld}
\end{figure}

\begin{figure}
\includegraphics[width=0.9\linewidth]{figures/cor_nswvic} \caption{Cross-correlation of six climate indices and rainfall in NSW/VIC study region.}\label{fig:cor-rain-nsw}
\end{figure}

Based on the correlation between the climatic indices and rainfall (as
shown in Figure \ref{fig:cor-rain-qld} and Figure
\ref{fig:cor-rain-nsw}), it can be concluded that:

\begin{itemize}
  \setlength{\itemsep}{0cm}
  \setlength{\parskip}{0cm}
  \item In QLD, the correlation between rainfall and SOI at zero time lags is the highest across all indices, outweighing the other ENSO indicators. IOD, had a weak influence in QLD.
  \item In NSW/VIC, again the SOI has the highest correlation with rainfall, followed by the IOD. Both occur at the zero time lags. 
  \item in both cases PDO had the weakest correlations, and this factor was not further considered as a predictor.
\end{itemize}

The above findings are consistent with previous studies. Although some
indices are serially correlated with rainfall up to several months, the
lag zero events have the most significant correlation coefficients.
Concurrent climatic index series were generally found most useful in
rainfall prediction (e.g. Risbey et al. 2009; Kamruzzaman, Beecham, and
Metcalfe 2011). The correlations between the climatic indices and
rainfall for each individual season have also been tested, and the
results were similar.

Rainfall in Australia shows strong seasonal patterns (Holper 2011;
Australian Bureau of Statistics 2012). For example, the north part of
the country is summer rainfall dominant with a dry winter, while most of
the southern part has a winter rainfall regime. This character is driven
by the movement of subtropical high pressure systems which dominate the
Australian climate (BoM 2012a). These characteristic summer and winter
rainfall patterns mean that the seasonal component of rainfall has a
periodic pattern which should be included in the model.

Long term trends in the regional rainfall in some parts of Australia are
significant (Hughes 2003; Gallant, Hennessy, and Risbey 2007; Chowdhury
and Beecham 2010). In the northern and eastern parts of the continent,
increasing rainfall is reported over the last century (Hughes 2003). The
presence of such long term trends may be confused with the outcome of a
step change in rainfall. As a result a linear trend term was implemented
in the model to remove any long term effect.

We assumed all the factors are additive smooth components in determining
rainfall following Kamruzzaman, Beecham, and Metcalfe (2011). Generally,
monthly rainfall has a skewed distribution so the normality assumption
of the residuals in a general linear model could be violated. In this
case, the rainfall model is a generalised additive model (GAM) (Hastie
and Tibshirani 1986) with a log link function \texttt{g()} and assuming
the residuals are gamma distributed (see Figure \ref{fig:hist-rain}).
This means all predictors are modelled as smooth functions, in this case
using the shrinkage version of the cubic regression splines (Wood 2011).
\vspace{0.5cm}

\begin{equation}
\begin{array}{lll}
g(E(\mathbf{R}_r)) = &\beta_0 + s_1(\mathbf{SOI}) + s_2(\mathbf{IOD}) + \\                &s_3(\mathbf{Nino3.4}) + s_4(\mathbf{Nino4}) + s_5(\mathbf{Season}) + \\
           &\beta_1\mathbf{Trend} + \boldsymbol{\epsilon}_r
\end{array}
\label{eq:model}
\end{equation}

The bold letters represent the time series vectors. The region is
indicated by \(r\), while \(\beta_u\) (\(_u\)=0, 1) are the fitted
coefficients in the model. \(s_v\) (\(_v\)=1, 2, 3) are the smooth
penalized cubic regression spline functions on the climatic indices and
the season. Apart from dropping the PDO as a predictor, all other
climate indices were included, allowing the model to select the
appropriate predictors.

A possible linear long term trend in the rainfall data is modelled by
\textbf{Trend} = 1,2,3\ldots{}n, where n is the total number of months
in the time series. \textbf{Season} is the seasonal component. The
climatic terms are also modelled with smooth functions. The effect of
large scale drivers on Australian rainfall is more likely to be seasonal
(Murphy and Timbal 2008; Schepen, Wang, and Robertson 2012), and the
smooth spline function is more flexible in reproducing the variability
in impacts of the climatic indices.

\begin{figure}
\includegraphics[width=0.9\linewidth]{figures/hist_rainfall} \caption{Distribution of monthly rainfall in (a) QLD and (b) NSW/VIC. Using a Kolmogorov-Smirnov test with shape = 1 and 2.4  for QLD and NSW/VIC respectively, rainfall in both regions can be modelled as a gamma distribution.}\label{fig:hist-rain}
\end{figure}

\subsection{Tree cover change as factor variable}

One of the main difficulties in empirical observation studies on the
effect of land cover change on rainfall is the lack of continuous
monitoring of land surface variables, or even, that no specific variable
can be defined that can clearly represent the land surface process.
Given the lack of a full picture of the land surface process, a factor
variable was used in this study to represent the abrupt land surface
change (see Equation \eqref{eq:model}). The change could be a result of
either land clearing or bush fires as long as it is permanent or takes a
long time to recover. Here we approached the problem with two different
models.

In the first method, the tree cover change was used as a predictor in
the regression model, represented by a factor variable \textbf{LC}. The
significance of the coefficient of \textbf{LC}, denoted as \(\beta'_5\)
in Equation \eqref{eq:model}, can be determined by a ratio test.

\begin{equation}
  \mbox{LC} = \left\{
  \begin{array}{ll}
     \mbox{Trees} \\
     \mbox{Removed}
  \end{array} \right.\\
  \label{eq:dummyC}
\end{equation}

Therefore in both regions, land cover is ``trees'' for the period before
land cover change and ``removed'' for the period after the change. Here
we simply assumed that vegetation cover change has occurred on every
pixel. The remaining term \(\epsilon_r\) is the amount of rainfall that
is attributed to other unspecified factors and random errors. Hence the
regression model becomes \vspace{0.5cm}

\begin{equation}
\begin{array}{lll}
g(E(\mathbf{R}_r)) = &\beta'_0 + s'_1(\mathbf{SOI}) + s'_2(\mathbf{IOD}) + \\
  &s'_3(\mathbf{Nino3.4}) + s'_4(\mathbf{Nino4}) + s'_5\mathbf{Season} + \\
  &\beta'_1\mathbf{Trend} + \beta'_2\mathbf{LC} + \boldsymbol{\epsilon'}_r
\end{array}
  \label{eq:model2}
\end{equation}

One of the difficulties is to point an exact time to the changes in the
vegetation cover in the two regions. In the QLD region, no exact time
can be assigned to the land clearing. According to the SLATS reports,
the most substantial clearing occurred between 2003 - 2004 . However,
the information on the change in type of land cover during the time
period is missing. Therefore, four scenarios were initially tested in
the analysis. In these scenarios the ``after change'' period started
from: (1) June 2003, (2) January 2004, (3) June 2004 and (4) January
2005. A further complicating factor is the influence of the ``millenium
drought'' over the study period and in particular to change to wet
conditions in 2010 - 2011 (Dijk and Viney 2013). In the NSW/VIC region,
severe bush fires were reported in early January 2003. Hence the
``tree'' cover state was up to December 2002 then it was changed to
``removed'' state from January 2003. However, there is similar
uncertainty about when the tree cover had recovered from the bushfire
and how this interacted with climate variability. As a result several
scenarios were tested for both locations: 1) analysis up to 7 years post
change (assumed to be 2010 for both regions); and 2) analysis of the
full period (2015). More detail about this is later in the methods. As a
starting date, the regression model was run from 1979 for both regions.

\subsection{Step trend test}\label{step-trend-test}

To support the regression analysis, a step trend test was used to detect
changes in rainfall as a result of vegetation cover change. This
nonparametric statistical test was modified from the Mann-Whitney
Rank-Sum test by (Hirsch and Gilroy 1985) and can identify a step change
in data which is cross correlated. The gridded rainfall dataset used in
this study has a high spatial correlation between neighbouring pixels
due to the underlying interpolation method. The advantages of using the
Rank-Sum test therefore are: (1) it does not depend on assumptions of
the data distribution; (2) it is not restricted to datasets with no
missing data; (3) it is robust and not as easily influenced by outliers
and negative numbers (Hirsch and Gilroy 1985).

The rainfall residuals from the regression model in Equation
\eqref{eq:model2} were used for the step trend test. According to (Hirsch
and Gilroy 1985), to detect a step change, using deseasonalised and
detrended data is important. Furthermore, as rainfall can only be
partially attributed to local sources and conditions, other effects are
introduced by large scale dynamics and changes in other climatic
factors. The assumption is that the regression model should remove these
effects, and additionally deseasonalise and detrend the rainfall data.
As a result, the local landuse effects are amplified in the variation of
the remaining residuals. The test, described in the following sub
section, subsequently associates trends in the rainfall residuals with
tree cover changes.

\subsubsection{Mann-Whitney rank-sum
statistic}\label{mann-whitney-rank-sum-statistic}

As indicated, the step trend test is a modified version of the
Mann-Whitney rank-sum statistic (Hirsch and Gilroy 1985). As a
nonparametric rank-based test, the Mann-Whitney test does not use the
exact values of rainfall but depends on the ranks of the data. For each
month, rainfall residuals of each year were ranked in an ascending
order. The ranking of January rainfall in a sample pixel k in QLD is
illustrated below:

\begin{longtable}[]{@{}ccc@{}}
\caption{Example of ranking rainfall residuals}\tabularnewline
\toprule
Year & Rainfall residuals & Rank \(R'_{1k}\)\tabularnewline
\midrule
\endfirsthead
\toprule
Year & Rainfall residuals & Rank \(R'_{1k}\)\tabularnewline
\midrule
\endhead
1998 & -0.3 & 6\tabularnewline
1999 & -60.9 & 2\tabularnewline
2000 & -16.1 & 4\tabularnewline
2001 & -71.7 & 1\tabularnewline
2002 & 111.1 & 7\tabularnewline
2005 & -7.2 & 5\tabularnewline
2006 & -60.5 & 3\tabularnewline
\bottomrule
\end{longtable}

\noindent Therefore, the smallest or most negative value has rank 1 and
the largest value has the maximum rank.

The before and after period in the data formed two groups of samples.
The split point of the two periods was based on the timing of the
vegetation cover changes. In the QLD region, changes occurred anytime
during 2003 and 2004. In contrast to the previous method, the time
period covering the land cover change was excluded, as the nonparametric
test allows missing data. Hirsch and Gilroy (1985) also pointed out that
the power of the test is higher if the data of the change period is
ignored. Hence 2003 and 2004 were excluded from the analysis. As a
result, the after-change period was 2005 - 2015 for the Queensland
location.

In the case of NSW/VIC, the bushfires broke out in early January 2003.
The change was within a relatively short period of the year. Therefore
the after change period in this region still started in January 2003.
Following Hirsch and Gilroy (1985), the before period was set to five
years (1998 - 2002) in both regions. The length of the after period is
difficult for the NSW/VIC as the regrowth would at some point have
impacted the local effects.

There is a further complication in the data. The year 2009 was the final
year of the millenium drought (Dijk and Viney 2013), followed by two
very wet years (also visible in Figure \ref{fig:ts-mean} in 2010 and
2011. As highlighted earlier, this means the the length of after period
could influence the analysis by the inclusion of more or fewer years. as
a result the period up to (and including) 2009 was compared to the full
period up to 2015. Further analysis up to 2010 is included in the
supplementary material.

The rank of rainfall in month j year i in pixel k is denoted as
\(R'_{ijk}\). The sum of ranks of rainfall in month j in pixel k before
the known intervention is:

\begin{equation}
  W_{jk} = \sum_{i=1}^{n_1}R'_{ijk}.
  \label{eq:Wj}
\end{equation}

\(n_1\) is the number of years before the land cover change. The
expected value of \(W_{jk}\) is

\begin{equation}
  \mu_w=n_1(n_1+n_2+1)/2
\end{equation}

\(n_2\) is the number of years after the change. Hence the expected
value of the rank sum before the intervention is the same for all months
and all pixels. The sum of ranks for the whole time period is fixed, as
\((n_1+n_2)(n_1+n_2+1)/2\). In this study, since there are only two
groups (before and after), knowing the rank-sum of one group is the same
as knowing the rank-sum of the other group. If the rainfall data is
temporally and spatially independent, the variance of \(W_{jk}\) is

\begin{equation}
  \sigma^2_w = n_1\cdot n_2(n_1+n_2+1)/m
\end{equation}

where m is the number of months which is 12 in the case of a full year.

\subsubsection{Step trend test}\label{step-trend-test-1}

Instead of completing the Mann-Whitney U-test, Hirsch and Gilroy (1985)
applied a standard normal Z test to the rank-sum statistics. As
highlighted, this modified test accounts for serial and cross
correlation in the data. In the case here, the deseasonalised and
detrended data shows little autocorrelation in the time series but
possesses strong cross correlation between neighbouring pixels, i.e.
\(R>0.99\).

The sum of \(W_{jk}\) for a block of \(ns\) pixels over the whole year,
\(\sum_{j=1}^{12}\sum_{k=1}^{ns}W_{jk}\), has mean

\begin{equation}
  E(\sum_{j=1}^{12}\sum_{k=1}^{ns}W_{jk})=12\cdot ns\cdot\mu_W
\end{equation}

and variance

\begin{equation}
  Var(\sum_{j=1}^{12}\sum_{k=1}^{ns}W_{jk})=\sum_{j=1}^{12}\sum_{k=1}^{ns}\sum_{h=1}^{ns}C(W_{jk},W_{jh}).
\end{equation}

\(C(W_{jk},W_{jh})\) is the covariance of the W statistics between pixel
k and pixel h in month j. When \(k=h\), \(C(W_{jk},W_{jh})=\sigma^2_w\).
When \(k\neq h\),

\begin{equation}
  C(W_{jk},W_{jh})=\sigma^2_w r(R_k,R_h)
\end{equation}

where \(r(R_k,R_h)\) is the product moment correlation coefficient of
the concurrent ranks in pixel k and h. Here \(r\) is calculated on the
full time series in each pixel. In the analysis, the test was applied to
a square block of four pixels each time. As argued by Hirsch and Gilroy
(1985), \(ns=4\) is the most optimal solution to balance the cost and
the gain in the test power.

The statistic of the step trend test is then defined as

\begin{equation}
  Z'=\frac{\sum_{j=1}^{12}\sum_{k=1}^{ns}W_{jk}-12\cdot ns\cdot\mu_w}{\sqrt{Var(\sum_{j=1}^{12}\sum_{k=1}^{ns}W_{jk})}}.
  \label{eq:Z}
\end{equation}

The above statistic is written for a 12 month period. By changing the
value 12, it can also be used to test seasonal rainfall change or for
other customized periods.

The null hypothesis (\(H_0\)) in this study is that there was no change
in rainfall due to land surface intervention. The results of the step
trend test can be interpreted according to the sign of the Z' score (see
Table \ref{tab:Zscore} (Chapter 23, P887 Hipel and McLeod 1994)). Z' is
normally distributed similar to the standard normal statistics Z. Hence
it can be compared to a standard normal distribution to determine the p
value.

\begin{longtable}[]{@{}c@{}}
\caption{The interpretation of Z' score in the step trend
test}\tabularnewline
\toprule
Z' \(>\) 0 \& rainfall decreases after change\tabularnewline
Z' \(<\) 0 \& rainfall increases after change\tabularnewline
Z' = 0 \& rainfall does not change\tabularnewline
\bottomrule
\end{longtable}

\section{Results}\label{results}

\subsection{Tree cover change}\label{tree-cover-change}

The pixels, where the tree cover change based on the MOD44B data was
significant (\(p \leq 0.05\)) in each study region, are shown in Figure
\ref{fig:tctrend} for the time series to 2009 (left panels) and the time
series to 2015 (right panels). In both cases it can be seen that the
area of negative cover change was greater in the series to 2009 than in
the series to 2015. But even the longer series indicates a large change
in the NSW/VIC region (bottom row). In the NSW/VIC region, much of the
tree loss between 2002 and 2003 was concentrated in the Snowy Mountains
close to the border of NSW and VIC. Tree cover loss occurred in large
parts of the QLD region between 2002 and 2005, but this tree loss was
spatially less concentrated. Most of the clearings appear in the centre
of this region in the series to 2009. The tree cover change map
(\ref{fig:tctrend} is consistent with the annual mean EVI trend map
(based on DLCD data, not shown here). However in the series to 2015 the
tree cover loss seems to have dissappeared (top right Figure
\ref{fig:tctrend}) and suggests an increase in tree cover over the area.
The implications of this will be discussed in more detail in the
discussion.

\begin{figure}
\includegraphics[width=0.9\linewidth]{figures/tc_figs} \caption{The maps show signifcant changes in tree cover identified from the MOD44B data between 2003 and 2009 (top left) and 2003 and 2015 (top right) in the NSW/VIC region and the Qld region (bottom left to 2009 and bottom right to 2015). The amount of change was calculated as the difference in tree cover before and after the specified land cover intervention and it is shown as the percentage of the ground area. Green colour indicates an increase in tree cover, while red colour indicates a decrease in tree cover.}\label{fig:tctrend}
\end{figure}

\subsection{Regression Model and significance of Vegetation Cover
Changes}\label{regression-model-and-significance-of-vegetation-cover-changes}

Generally the regression model only explains a limited amount of the
rainfall variability. The model in Equation \eqref{eq:model2} accounts for
around
15\%\footnote{Here the adjusted $R^2$ was reported. Adjusted $R^2$ is the coefficient of determination, a measurement of the amount of variability predicted by the model adjusting for the number of explanatory terms}
of the rainfall variation in both regions. The residual analysis shows
that the assumptions of the regression model are generally met (Figure
\ref{fig:residuals}. The standardised residual plots, however, show some
funnelling for the NSW/VIC regions, suggesting non constant variance.
The residual patterns are consistent for all pixels within each region.

The model, however, confirms the importance of the climate drivers and
the seasonality in Australian rainfall. Even at the grid level, the
seasons and several of the climatic indices were significant
(\(p \leq 0.05\)) everywhere in both regions. The explaining power of
the model is mostly due to these variables. The climate drivers (at lag
zero) accounted, on average, for 6.7\% of the rainfall variability in
both the QLD region and the NSW/VIC region (see Figure \ref{fig:rsq} for
the distribution of \(R^2\) in these two regions). These figures are
within the upper bound of seasonal rainfall predictability by a SST
anomaly field reported by (Westra and Sharma 2010).

\begin{figure}
\includegraphics[width=0.9\linewidth]{figures/gam_check} \caption{The residual analysis of a sample pixel in the QLD region (top) and NSW/VIC region (bottom).}\label{fig:residuals}
\end{figure}

\begin{figure}
\includegraphics[width=0.9\linewidth]{figures/soi_explhistogram} \caption{The performance of the regression model if rainfall is only modelled by the climate drivers. It shows the percentage of rainfall variability that can be explained by the climate drivers for the Qld and NSW/VIC region}\label{fig:rsq}
\end{figure}

Statistically significant long term trends were only observed in part of
the NSW region (results not shown). However, this result might prove or
disprove the existence a long term trend in rainfall. The overall time
period is fairly short (Koutsoyiannis 2006) and more pixels in NSW/VIC
could indicate a significant step change if the long term trend effect
is not removed by the model. As trend free data is an important
requirement for the step trend test, the trend term was kept in the
regression model to ensure the detection of step change was not due to a
possible long term trend.

\begin{figure}
\includegraphics[width=0.9\linewidth]{figures/Cp_30yrs} \caption{The spatial distribution of significance of land cover step change variable in the GAM model predicting changes in rainfall in the both regions. The top row reflects NSW/VIC, while the bottom relates to the Qld study site. The left hand plots are the data up to 2009, while the right hand plots are for the full series. The p value reported is for the land cover variable in the model.}\label{fig:LCp}
\end{figure}

The land cover variable in the model (Equation \eqref{eq:model2}) aims to
identify a step change in the rainfall before and after the observed
change in land cover. The variable was only significant
(\(p \leq 0.05\)) for the rainfall estimates in some areas in NSW/VIC,
as shown in Figure \ref{fig:LCp}. However, the number of pixels where
the landcover variable was significant was much greater for the series
up to 2009, compared to the series to 2015. There was no direct
relationship between the areas of bushfires in Figure
\ref{fig:bushfire}. No significant step change due to the land cover
changes was found in rainfall in the QLD region in the series to 2015,
but a small area of change was identified in the period to 2009 (left
bottom panel).

Figure \ref{fig:LCp} for NSW/VIC suggests that a significant step change
in rainfall related to land cover change was found in an area larger
than where the bushfires has occurred. This might be showing a large
scale effect that reaches beyond the vegetation cover change effect.

More generally, the model suggests that the tree cover has a positive
impact on rainfall in both regions. The fitted coefficients for the Land
cover change variable were consistently positive for the ``tree'' part
of the series. It implies that rainfall was higher when the surface was
covered by trees.

\subsection{Step Trend Test}\label{step-trend-test-2}

\begin{figure}
\includegraphics[width=0.9\linewidth]{figures/step_new} \caption{Spatial distribution of the step trend test Z' statistics in the two study sites. Panels on the top are for the NSW/Vic site, while panels of the bottom ar for Qld. The left panels are for the series to 2009, while the right panels are for the full data series. Warm colours (yellow, orange and red) are for positive Z' values which indicate decreasing rainfall trend due to the land surface intervention. Cold colours (light blue to blue) are for negative Z' values which indicate increasing rainfall trend. The deeper the colour, the more significant the statistic.}\label{fig:steptest30}
\end{figure}

The spatial step trend test Z' scores for the Qld region (top row) and
NSWVIC region (bottom row) are shown in Figure \ref{fig:steptest30}. The
figure also contains the period to 2009 (left panels) and the period to
2015 (right panels). This figure provides two types of information: the
sign and the significance level. The sign indicates the direction of the
step change, as listed in Table \ref{Zscore}. In each region up to 2009,
there is a broad area of positive Z' values which implies a decrease in
rainfall. There appears to be little relationship between the locations
where changes in tree cover are observed (Figure @ref(fig:tc\_trend))
and the patterns in the Z' score. However, there is not necessarily a
direct relationship as movement of air masses could mean that actual
changes of rainfall are observed close by, but not necessarily exactly
at areas with changes of landcover. There is once again a difference
between the left panels and the right panels, indicating that
considering a longer time series (to 2015) reduces the pattern and
significance of Z' scores. In the QLD region to 2009, 21\% of the pixels
obtained a positive Z' score with p \(<\) 0.1, and for the period to
2015 no pixels with positive Z' score at p \(<\) 0.1 occurred. In the
NSW/VIC region to 2009, only 2.6\% of pixels in the alpine area having a
positive Z' score with p \(<\) 0.1. In general it is only a small
proportion of both study regions.

\begin{figure}
\includegraphics[width=0.9\linewidth]{figures/ResidualBoxplotchange} \caption{Boxplots of annual rainfall residuals (estimated based on Equation 2 before and after the land cover intervention during 1979 - 2015 in the study regions. On average, the after period has a significantly lower annual rainfall residual in NSW/VIC, but a signifcantly higher annual rainfall residual in the Qld study area}\label{fig:meandiff}
\end{figure}

The rainfall residuals of the two periods (before-change since 1979 and
after-change) were also compared (Figure \ref{fig:meandiff}) using a
simple t-test. From the boxplots it is difficult to see that there were
significantly different (\(p < 0.05\)) mean values between the
``before'' and ``after'' periods in both regions. For the Queensland
locations, there was no significant difference if measured to 2009 and
slightly less rain (\(p < 0.05\)) before if measured to 2015
(\(~ 2 mm/month\)). In contrast, for the NSW/VIC locations there was
also slightly more rain (\(p < 0.05\)) after the change if measured to
2015 (\(~ 1 mm/month\)), but significantly less rain (\(p < 0.05\))
after the change (\(~ 9 mm/month\)) if measured to 2009.

A closer look at the rainfall data indicates that rainfall was
consistently very low in 2006 for both regions and consistently high for
2010 \ref{fig:ts-mean}. The low rainfall in 2006 was due to a weak El
Niño and unrecovered conditions from the previous drought, in contrast,
2010 brough record drought breaking rains across Eastern Australia
related to a La Niña (Dijk and Viney 2013). While the regression model
has removed most of the effect of ENSO, the effect of these anomolous
years could still be visible in the residuals if the response of the
rainfall to the ENSO effect is non-linear, and for example the memory of
the past drought is persistent. This can also explain the significant
results in the previous method. To remove the possible outlier effects
from 2006 and 2010, the two periods were compared excluding the 2006 and
2010 rainfall. In addition, the group of pixels showing a negative step
change in rainfall (\(p < 0.10\)) and the group of pixels indicating no
change were analysed separately for the period to 2009. An unpaired
unequal variance two sample t-test was applied to test whether the after
period had lower mean rainfall than the before period, excluding the
2006 and 2010 years for the full period, and excluding 2006 for the
period to 2009.

For the Queensland locations using the regression model residuals and
excluding 2006 and 2010, there was no difference between the period
before and after the landuse change for the full data series. However
for the period to 2009, there is slightly more rain
(\(p < 0.05, ~ 2mm/month\)) before than after the change. For the
NSW/VIC locations for the same analyses, there is slightly more rain
(\(p < 0.05\)) after the change if measured to 2015 (\(~ 2 mm/month\)),
and for the period to 2009 there is still significantly less rain
(\(p < 0.05, ~ 5 mm/month\)). Focussing on the pixels with a negative
step change (positive significant Z' score) and the monthly rainfall
totals, only the period to 2009 is of interest, as the full period had
too few significant pixels (\ref{fig:steptest30}). For both regions and
excluding 2006, the Qld area had significantly more rain before than
after at \(p < 0.05\). For the NSW/VIC area, there was similarly
significantly more rain before than after the change
(\(p < 0.05 and 100mm/month\)). For pixels where the Z' score was not
significantly positive (\(p > 0.1\) and Z' score negative), the Qld area
showed no significant difference in monthly rainfall totals before and
after. For the NSW/VIC area there was still significantly less rain
after the change (\(p < 0.05\)) but difference in monthly rainfall
totals was smaller than for positive and signifcant Z' scores: 24
mm/month less.

The choice of \(ns\) has some impact on the test results (Hirsch and
Gilroy 1985). The cases of \(ns = 1\) and \(ns = 9\) were also tested.
The results for \(ns = 1\) showed a lower number of significant
pixels(at p \(<\) 0.10) compared to the \(ns = 4\) test, with only 66
pixels in the Queensland area in the period to 2010 and a slightly
higher 6 pixels in the period to 2015. No pixels were significant at p
\(<\) 0.1 in the NSW/VIC area. The results from the \(ns = 9\) test was
not different from the \(ns = 4\) test. The power of the test does not
change much after \(ns = 4\), as shown by Hirsch and Gilroy (1985).

As part of the analysis, the ``field significance'' of the Z' score test
was considered to better interpret the step change at regional scales
from multiple local tests (Wilks 2006, Westra, Alexander, and Zwiers
(2013)). Here, the bootstrapping resampling method from Westra,
Alexander, and Zwiers (2013) was used to evaluate the field significance
for the period up to 2009 (as the period to 2015 essentially showed no
significant positive Z' scores). This means the spatial structure of the
pixels was maintained, but the order of the years and months was changed
by random resampling. For each resampling the test statistics identifies
the percentage of the pixels with significant positive step change for
the step trend test. The test statistics on 1000 resampled replicates
were used to develop the distribution of these percentage values under
the local null hypothesis that there was no step change. The results
showed that none of the resampled series indicated any significant step
trend for both regions, suggesting that the actual values found are for
a unique series of years.

\section{Discussion}\label{discussion}

\begin{table}

\caption{\label{tab:summarytable}Summary table of all tests on the two regions and for the two time periods}
\centering
\resizebox{\linewidth}{!}{
\begin{tabular}[t]{lllll}
\toprule
Test & Qld 2009 & Qld 2015 & NSW/VIC 2009 & NSW/VIC 2015\\
\midrule
LC variable & small area on west side & no pixels significant & Large area & Smaller area than 2009\\
t-test regression \textbackslash{}\& model residuals & not significant & less rain before & less rain after & slightly less rain after\\
t-test residuals \textbackslash{}\& excluding 2006 and 2010 & slightly more rain after & no significant difference & significantly less rain after & slightly more rain after\\
Z' score & 21\% of pixels at p < 0.1 & 1 pixel at p < 0.1 & 2.6\% of pixels at p < 0.1 & no pixels at p < 0.1\\
t-test pixels positive Z' score & significantly less rain after & NA & 100 mm/month less rain after (p < 0.05) & NA\\
t-test other pixels & no difference & NA & 24 mm/month less rain after (p < 0.05) & NA\\
\bottomrule
\end{tabular}}
\end{table}

Overall the summary table of the results indicates that the effects of
land clearing or bushfire on rainfall are not easily detected, even with
a range of different statistical tests. In addition several of the tests
appear to indicate contradictory results, in particular in the
comparison between the two regions and between the two time periods.

Generally, empirical studies on LCC-precipitation interaction are
conducted within an area with known land surface intervention (e.g.
Otterman et al. 1990,Durieux, Machado, and Laurent (2003),Negri et al.
(2004),Sato, Kimura, and Hasegawa (2007)). However, these locations are
rare and difficult to isolate from real landscape change. Modelling
studies are abundant, but are generally not linked to observed data. In
this study we therefore tested the effect of land cover change across a
broad region, rather than only for locations where changes were known to
occur or have occured. The advantage of the suggested approach is that
it does not require a long time series of land cover data as this is
usually unavailable. Furthermore, it does not assume a specific
relationship between vegetation cover change and rainfall but allows the
data to show this relationship, by applying the analysis to a broader
area outside the boundary of the vegetation cover change. This approach
is expected to provide a way to reduce the risk of a false positive
paradox, by comparing results between areas with and without vegetation
cover change.

While there is some indication that the observed landuse changes (Figure
\ref{fig:tctrend}) cause a decrease in the rainfall, our analysis has
not been able to give an unequivocal answer. There are several possible
complicating fatcors in the data that could explain the differences
between the two regions and the two time periods.

\subsection{Rainfall variability}\label{rainfall-variability}

The first and most significant effect on the results remains the
variability in the rainfall. The regression model used here is a simple
model. We only consider the important effects of a historical trend,
seasonality and climate drivers. The purpose of using the regression
model is to remove the year to year and month to month variability in
rainfall and therefore strengthen the land cover signal in the
residuals. However, the model shows that trend, seasonality, ENSO and
IOD together explain no more than 30\% of the rainfall variability, and
only around 7\% on average is due to the climate drivers (Figure
\ref{fig:rsq} . And while this is consistent with the literature e.g.
Westra and Sharma (2010), a large amount of variation is left in the
model residuals. Rainfall is generally considered a stochastic process
(e.g. Fowler et al. 2005,Cowpertwait, Salinger, and Mullen (2009),Burton
et al. (2010)) and clearly some of the variability is either a different
response to a combination of climate factors (as interactions were not
tested in the model), or a strongly non-stationary response to the
climate drivers. The remaining variability in the residuals increases
the difficulty to detect a change in rainfall. This is clearly
demonstrated in the difference in land cover variable significance and
the Z' score between the series to 2009 and the series to 2015. While
2009 was at the end of the drought, the series to 2015 includes the
record breaking year 2010 (Figure \ref{fig:ts-mean}).

The severe bushfires in 2003 were also triggered by the extreme drought
conditions during the millenium drought (Dijk and Viney 2013). Although
the drought on rainfall has been accounted for in part by the model, a
further delayed or cumulative impact of drought could be feeding into
the local land-atmosphere interaction. As a result, the rainfall
feedback to the vegetation cover change could be weak under the dry
conditions, and this could have affected the result.

\subsection{Vegetation dynamics}\label{vegetation-dynamics}

The second possible effect is the dynamic nature of the vegetation
clearing and recovery. Although land clearing has occurred at a high
rate and broad scale in Queensland (Department of Science, Information
Technology and Innovation 2017), the clearing does not have a clear
start and end point. QLD has a long history of land clearing. According
to the series of SLATS reports on land cover changes in QLD released by
the Queensland government, land clearing continued in and around the
study region between 1988 - 2008. Major broad scale and high rate
clearings occurred in 1999 - 2000 and 2002 - 2004 (Figure
\ref{fig:slat}). And even though there was a decrease in land clearing
post 2005, it is difficult to define a clear cut change in this region.
The more continuous ongoing land clearing could have reduced the
significance of a step change.

\begin{figure}
\includegraphics[width=0.9\linewidth]{figures/slats} \caption{Woody vegetation clearing rate in QLD for the two major bioregions in the study area. The data were obtained from SLATS (2017), and clearly indicate a sharp decrease in the clearing rate after 2005.}\label{fig:slat}
\end{figure}

The two approaches used in this study indicate quite different results
on the level of change in the NSW/VIC region. The regression model
showed that a large area in the NSW/VIC region has experienced a
significant land cover effect (\(p < 0.05\)) on the rainfall after 2003.
The effect in the step trend test (Z' scores) is much mroe subdued, even
though it was able to detect significant changes (\(p < 0.1\)) within
the area in the period to 2009, but almost no effect in the period to
2015. In contrast, the results are almost reversed in the Qld area.
While some of this could be due to the difficulty in removing year on
year rainfall variability, some of this might also be due to vegetation
regrowth in both locations.

The specific vegetation class in the Queensland area is well-known for
rapid regrowth and ``thickening'' in favourable conditions (Gowen and
Bray 2016), and this could explain the change in vegetation cover
between 2009 and 2015 in the Qld area (Figure \ref{fig:tctrend}). In
particular the favourable rainfall years of 2010 and 2011 would have
boosted regrowth, increasing evapotranspiration and therefore decreasing
the effect of the rainfall change.

The different causes of vegetation cover change in these two regions
could lead to different post-change characteristics. The magnitude of
EVI decreasing trends in the QLD region are less than in NSW region, as
reported in the DLCD data. This is due to the lower tree density in the
QLD region than in the NSW/VIC region before land surface interventions.
The significant bushfires (Figure \ref{fig:fig2bushfire}) would have
drastically reduced the vegetation cover and recovery was very slow in
some areas within the NSW/VIC area. The persistent drought in the 2000s
(Howden 2012,Dijk and Viney (2013)) delayed the regrowth of trees. On
the other hand, replacing tree cover with pasture and crops in the Qld
area might have a relatively subtle impact on the EVI.

\subsection{Misalignment rainfall effects and vegetation
clearing}\label{misalignment-rainfall-effects-and-vegetation-clearing}

Particularly in the NSW/VIC area, there is no direct overlap between the
pixels with significant vegetation change (Figure \ref{fig:tctrend}) and
the pixels with significant land cover variables or positive Z'scores
(Figure \ref{fig:LCp} and Figure \ref{fig:steptest30}). While this seems
possibly surprising at first, a plausible explanation could be the
general movement of moisture and climate systems over the landscape,
resulting in a possible shift of where the moisture is evaporated and
where rainfall occurs. Specifically for this reason, this study selected
rather large spatial boxes to capture the overall response rather than
the pixel by pixel comparison/

\subsection{Gridded monthly rainfall
data}\label{gridded-monthly-rainfall-data}

The rainfall data used in this study is a gridded data set. This data
set is robust and consistent over a long time series (from 1900 to
current) and has a broad national wide coverage which can provide more
information spatially. However, high cross correlation between pixels,
due to the interpolation method generating this data set, can also
introduce spatial noise. In the step trend test the cross correlation
was accounted for. Some other methods are also available which can be
used to perform a comparative trial. For example, (Narisma et al. 2007)
applied a spatial Gaussian filter on a similar data set and used wavelet
analysis to detect step change in rainfall. High quality station data is
another option to test whether the observed spatial pattern in the step
trend test results was not due to the gridded data itself. Resampling
methods, such as bootstrapping and permutation (Wilks 1997,Kundzewicz
and Robson (2004),Westra, Alexander, and Zwiers (2013)), can also be
used to further assess the strength of significance of results and
incorporate spatial and temporal patterns in the analysis. While the
gridded data set is most useful in regions with sparse rain gauge
networks, it can actually reduces information where the rain gauge
density is high (Jones, Wang, and Fawcett 2009). In the NSW/VIC area,
the coverage of rainfall stations is more intensive but they are mainly
located in the valleys. The interpolated data might not be the best
represention of the local rainfall.

\subsection{General approach}\label{general-approach}

Parametric tests are generally more powerful than nonparametric test in
detecting a trend, when the data is normally distributed (Onoz and
Bayazit 2003,Kundzewicz and Robson (2004)). As a non-parametric test,
the step trend test has the advantages of distribution free and having
no restriction on missing data (Hirsch and Gilroy 1985). This is
particularly useful in rainfall analysis since rainfall data is usually
skewed. On the other hand, the disadvantages of non-parametric tests,
such as being limited to hypothesis testing and weaker in power, also
hold for the step trend test (Whitley and Ball 2002).

Overall, the current study provides a clear approach build on several
lines of evidence to provide some evidence to reject the null hypothesis
(no step change in rainfall occurs as a result of tree cover loss).
Limited by the available data, the time frame under study was chosen to
include a long lasting drought period (Holper 2011, Dijk and Viney
(2013)). The strong impact of this prolonged drought might have
suppressed the land-atmosphere interaction and confused the cause and
effect relation between rainfall and vegetation cover change. This could
be one of the reasons that the LCC effects on the local climate found in
other studies (e.g. Görgen et al. 2006,McAlpine et al. (2007)) are not
found to be significant here. Possible future work could focus on a
non-drought period, or when a longer series of land cover data is
available. Wile the power of the test can be improved with the longer
length of the after-intervention period (Hirsch and Gilroy 1985), the
dynamic nature of vegetation regrowth in this case study prevents this
effect. A better approach might be to build a global study that
investigates multiple locations where drastic landcover changes have
taken place, which would also remove some of the climate variability
effects due to the larger sample size.

\section{Conclusions}\label{conclusions}

In this study, we present an approach to identify iffound some
observationl data based evidence, although not strong, that vegetation
cover change has changed local rainfall. The semi-parametric method and
non-parametric method did totally agree on detecting a significant step
change in rainfall in the hot dry QLD region where land clearing has
occurred. On the other hand, the bushfires in the humid, temperate
mountain range in the NSW/VIC region has experienced reduced rainfall.
But the dry spell also plays an important role in the results.

Drought has had a pronounced impact on the land surface condition during
the study period, leading to significant reduction in vegetation and
extreme events such as bushfires. The associated lack of rainfall and
high temperatures may mask the step change in the vegetation. Hence, the
signal of LCC feedback on rainfall is probably weaker under such
regional dry conditions, as the impact of LCC on rainfall is mainly
through changes in moisture convergence (Görgen et al. 2006,Pitman and
Hesse (2007)).

\section{acknowledgments}\label{acknowledgments}

CL was supported by an Australian Postgraduate Award for this work.

\newpage

\section{appendix}\label{appendix}

\subsection{Summary of Data}\label{summary-of-data}

\newpage

\section*{References}\label{references}
\addcontentsline{toc}{section}{References}

\hypertarget{refs}{}
\hypertarget{ref-ABARES2010}{}
ABARES. 2010. ``Land Use of Australia, Version 4, 2005/2006.''

\hypertarget{ref-ABC2003}{}
ABC News. 2003. ``Kosciuszko Slow to Recover from Bushfires.''

\hypertarget{ref-ABS2012}{}
Australian Bureau of Statistics. 2012. ``Geography and Climate -
Australia's Climate.''

\hypertarget{ref-Ben-Gai1998}{}
Ben-Gai, T., A. Bitan, A. Manes, P. Alpert, and S. Rubin. 1998.
``Spatial and Temporal Changes in Rainfall Frequency Distribution
Patterns in Israel.'' \emph{Theoretical and Applied Climatology} 61
(3-4): 177--90.
\href{\%3CGo\%20to\%20ISI\%3E://000078025000005}{\textless{}Go to ISI\textgreater{}://000078025000005}.

\hypertarget{ref-BoM2009}{}
BoM. 2009. ``Gridded Monthly Rainfall Metadata.'' Australian Bureau of
Meteorology.
\url{http://www.bom.gov.au/climate/austmaps/metadata-monthly-rainfall.shtml}.

\hypertarget{ref-BoM2012a}{}
---------. 2012a. ``Australia - Climate of Our Continent.''

\hypertarget{ref-BoM2012}{}
---------. 2012b. ``Australian Climate Influence.''

\hypertarget{ref-Bosilovich2006}{}
Bosilovich, Michael G., and Jiun-Dar Chern. 2006. ``Simulation of Water
Sources and Precipitation Recycling for the MacKenzie, Mississippi, and
Amazon River Basins.'' \emph{Journal of Hydrometeorology} 7 (3):
312--29. \url{http://dx.doi.org/10.1175\%2FJHM501.1}.

\hypertarget{ref-Burton2010}{}
Burton, A, HJ Fowler, S Blenkinsop, and CG Kilsby. 2010. ``Downscaling
Transient Climate Change Using a Neyman-Scott Rectangular Pulses
Stochastic Rainfall Model.'' \emph{Journal of Hydrology} 381 (1):
18--32.

\hypertarget{ref-Chagnon2005}{}
Chagnon, F. J. F., and R. L. Bras. 2005. ``Contemporary Climate Change
in the Amazon.'' \emph{Geophysical Research Letters} 32 (13).
\href{\%3CGo\%20to\%20ISI\%3E://WOS:000230640100004\%20http://www.agu.org/journals/gl/gl0513/2005GL022722/2005GL022722.pdf}{\textless{}Go to ISI\textgreater{}://WOS:000230640100004 http://www.agu.org/journals/gl/gl0513/2005GL022722/2005GL022722.pdf}.

\hypertarget{ref-Chowdhury2010}{}
Chowdhury, R. K., and S. Beecham. 2010. ``Australian Rainfall Trends and
Their Relation to the Southern Oscillation Index.'' \emph{Hydrological
Processes} 24 (4). John Wiley \& Sons, Ltd.: 504--14.
\url{http://dx.doi.org/10.1002/hyp.7504}.

\hypertarget{ref-Cowpertwait2009}{}
Cowpertwait, PSP, Jim Salinger, and B Mullen. 2009. ``A Spatial-Temporal
Stochastic Rainfall Model for Auckland City: Scenarios for Current and
Future Climates.'' \emph{Journal of Hydrology (NZ)} 48: 95--109.

\hypertarget{ref-Deo2009}{}
Deo, RC, JI Syktus, CA McAlpine, and KK Wong. 2009. ``The Simulated
Impact of Land Cover Change on Climate Extremes in Eastern Australia.''
In \emph{Proceedings of the 18th World Imacs Congress and Modsim09
International Congress on Modelling and Simulation}, 2035--41.
Modelling; Simulation Society of Australia; New Zealand Inc.;
International Association for Mathematics; Computers in Simulation.

\hypertarget{ref-SLATS2001}{}
Department of Natural Resources and Mines. 2005. ``Land Cover Change in
Queensland 2001-2003, Incorporating 2001-2002 Adn 2002-2003 Change
Periods: A Statewide Landcover and Trees Study (SLATS) Report.''
Brisbane: Department of Natural Resources; Mines.

\hypertarget{ref-SLATS2004}{}
Department of Natural Resources and Water. 2007. ``Land Cover Change in
Queensland 2004-2005: A Statewide Landcover and Trees Study (SLATS)
Report.'' Brisbane: Department of Natural Resources; Water.

\hypertarget{ref-SLATS2017}{}
Department of Science, Information Technology and Innovation. 2017.
``Land Cover Change in Queensland 2015-2016: A Statewide Landcover and
Trees Study (SLATS) Report.'' Brisbane: Department of Science,
Information Technology; Innovation.

\hypertarget{ref-vanDijk2013}{}
Dijk, Beck van, Albert I. J. M., and Neil R. Viney. 2013. ``The
Millennium Drought in Southeast Australia (2001--2009): Natural and
Human Causes and Implications for Water Resources, Ecosystems, Economy,
and Society.'' \emph{Water Resources Research} 49 (2): 1040--57.
doi:\href{https://doi.org/10.1002/wrcr.20123}{10.1002/wrcr.20123}.

\hypertarget{ref-Dirmeyer2009}{}
Dirmeyer, P. A., K. L. Brubaker, and T. DelSole. 2009. ``Import and
Export of Atmospheric Water Vapor Between Nations.'' \emph{Journal of
Hydrology} 365 (1-2): 11--22.
\href{\%3CGo\%20to\%20ISI\%3E://000263575600003}{\textless{}Go to ISI\textgreater{}://000263575600003}.

\hypertarget{ref-Durieux2003}{}
Durieux, Laurent, Luiz Augusto Toledo Machado, and Henri Laurent. 2003.
``The Impact of Deforestation on Cloud Cover over the Amazon Arc of
Deforestation.'' \emph{Remote Sensing of Environment} 86 (1): 132--40.
\url{http://www.sciencedirect.com/science/article/pii/S0034425703000956}.

\hypertarget{ref-Eltahir1996}{}
Eltahir, E. A. B., and R. L. Bras. 1996. ``Precipitation Recycling.''
\emph{Reviews of Geophysics} 34 (3): 367--78.
\href{\%3CGo\%20to\%20ISI\%3E://A1996VD58700003}{\textless{}Go to ISI\textgreater{}://A1996VD58700003}.

\hypertarget{ref-Fowler2005}{}
Fowler, HJ, CG Kilsby, PE O'connell, and A Burton. 2005. ``A
Weather-Type Conditioned Multi-Site Stochastic Rainfall Model for the
Generation of Scenarios of Climatic Variability and Change.''
\emph{Journal of Hydrology} 308 (1): 50--66.

\hypertarget{ref-Gaertner2001}{}
Gaertner, M. A., O. B. Christensen, J. A. Prego, J. Polcher, C.
Gallardo, and M. Castro. 2001. ``The Impact of Deforestation on the
Hydrological Cycle in the Western Mediterranean: An Ensemble Study with
Two Regional Climate Models.'' \emph{Climate Dynamics} 17 (11): 857--73.

\hypertarget{ref-Gallant2007}{}
Gallant, Ailie J. E., Kevin J. Hennessy, and James Risbey. 2007.
``Trends in Rainfall Indices for Six Australian Regions: 1910 - 2005.''
\emph{Australian Meteorological Magazine} 56: 223--39.

\hypertarget{ref-Gimeno2010}{}
Gimeno, Luis, Anita Drumond, Raquel Nieto, Ricardo M. Trigo, and Andreas
Stohl. 2010. ``On the Origin of Continental Precipitation.''
\emph{Geophysical Research Letters} 37 (13). AGU.
\url{http://dx.doi.org/10.1029/2010GL043712}.

\hypertarget{ref-Gowen2016}{}
Gowen, Rebecca, and Steven G. Bray. 2016. ``Bioeconomic Modelling of
Woody Regrowth Carbon Offset Options in Productive Grazing Systems.''
\emph{The Rangeland Journal} 38 (3): 307--17.
doi:\href{https://doi.org/https://doi.org/10.1071/RJ15084}{https://doi.org/10.1071/RJ15084}.

\hypertarget{ref-Gorgen2006}{}
Görgen, K., A. H. Lynch, A. G. Marshall, and J. Beringer. 2006. ``Impact
of Abrupt Land Cover Changes by Savanna Fire on Northern Australian
Climate.'' \emph{Journal of Geophysical Research-Atmospheres} 111 (D19):
D19106. \url{http://dx.doi.org/10.1029/2005JD006860}.

\hypertarget{ref-Hastie1986}{}
Hastie, Trevor, and Robert Tibshirani. 1986. ``Generalized Additive
Models.'' \emph{Statistical Science}, 297--310.

\hypertarget{ref-Hipel1994}{}
Hipel, K.W., and A.I. McLeod. 1994. \emph{Time Series Modelling of Water
Resources and Environmental Systems}. Vol. 45. Elsevier Science Ltd.

\hypertarget{ref-Hirsch1985}{}
Hirsch, Robert M., and Edward J. Gilroy. 1985. ``Detectability of Step
Trends in the Rate of Atmospheric Deposition of Sulfate.'' \emph{JAWRA
Journal of the American Water Resources Association} 21 (5): 773--84.
\url{http://dx.doi.org/10.1111/j.1752-1688.1985.tb00171.x}.

\hypertarget{ref-Holper2011}{}
Holper, Paul N. 2011. \emph{Climate Change, Science Information Paper:
Australian Rainfall-Past, Present and Future}. Canberra: CSIRO.

\hypertarget{ref-Howden2012}{}
Howden, Saffron. 2012. ``It's Official: Australia No Longer in
Drought.'' \emph{Brisbane Times}.

\hypertarget{ref-Hughes2003}{}
Hughes, Lesley. 2003. ``Climate Change and Australia: Trends,
Projections and Impacts.'' \emph{Austral Ecology} 28 (4). Blackwell
Science Pty: 423--43.
\url{http://dx.doi.org/10.1046/j.1442-9993.2003.01300.x}.

\hypertarget{ref-Jones2009}{}
Jones, D.A., W. Wang, and R. Fawcett. 2009. ``High-Quality Spatial
Climate Data-Sets for Australia.'' \emph{Australian Meteorological and
Oceanographic Journal} 58 (4): 233--48.

\hypertarget{ref-Junkermann2009}{}
Junkermann, W., J. Hacker, T. Lyons, and U. Nair. 2009. ``Land Use
Change Suppresses Precipitation.'' \emph{Atmospheric Chemistry and
Physics} 9 (17): 6531--9.
\href{\%3CGo\%20to\%20ISI\%3E://WOS:000269778500018}{\textless{}Go to ISI\textgreater{}://WOS:000269778500018}.

\hypertarget{ref-Kamruzzaman2011}{}
Kamruzzaman, M., S. Beecham, and A. V. Metcalfe. 2011.
``Non-Stationarity in Rainfall and Temperature in the Murray Darling
Basin.'' \emph{Hydrological Processes} 25 (10). John Wiley \& Sons,
Ltd.: 1659--75. \url{http://dx.doi.org/10.1002/hyp.7928}.

\hypertarget{ref-Koutsoyiannis2007}{}
Koutsoyiannis, Demetris. 2006. ``Nonstationarity Versus Scaling in
Hydrology.'' Journal Article. \emph{Journal of Hydrology} 324 (1-4):
239--54.
\href{http://www.sciencedirect.com/science/article/B6V6C-4HP6GJ9-1/2/67dcc2e8f8e75b7abb918617f5a79a37\%20}{http://www.sciencedirect.com/science/article/B6V6C-4HP6GJ9-1/2/67dcc2e8f8e75b7abb918617f5a79a37}.

\hypertarget{ref-kucharski_further_2013}{}
Kucharski, Fred, Ning Zeng, and Eugenia Kalnay. 2013. ``A Further
Assessment of Vegetation Feedback on Decadal Sahel Rainfall
Variability.'' \emph{Climate Dynamics} 40 (5-6): 1453--66.
\url{http://link.springer.com/article/10.1007/s00382-012-1397-x}.

\hypertarget{ref-Kuzcera1987}{}
Kuczera, George. 1987. ``Prediction of Water Yield Reductions Following
a Bushfire in Ash-Mixed Species Eucalypt Forest.'' \emph{Journal of
Hydrology} 94 (3-4): 215--36.
doi:\href{https://doi.org/https://10.1016/0022-1694(87)90054-0}{https://10.1016/0022-1694(87)90054-0}.

\hypertarget{ref-Kundzewicz2004}{}
Kundzewicz, Z.W., and A.J. Robson. 2004. ``Change Detection in
Hydrological Records-a Review of the Methodology/Revue Méthodologique de
La Détection de Changements Dans Les Chroniques Hydrologiques.''
\emph{Hydrological Sciences Journal} 49 (1).

\hypertarget{ref-Lymburner2010}{}
Lymburner, Leo, Peter Tan, Norman Mueller, Richard Thackway, Adam Lewis,
Medhavy Thankappan, Lucy Randall, Anisul Islam, and Udaya Senarath.
2010. ``250 Metre Dynamic Land Cover Dataset.'' Geoscience Australia,
Canberra.

\hypertarget{ref-Ma2011}{}
Ma, H. Y., C. R. Mechoso, Y. Xue, H. Xiao, C. M. Wu, J. L. Li, and F. De
Sales. 2011. ``Impact of Land Surface Processes on the South American
Warm Season Climate.'' \emph{Climate Dynamics} 37 (1-2): 187--203.
\href{\%3CGo\%20to\%20ISI\%3E://WOS:000293403000012}{\textless{}Go to ISI\textgreater{}://WOS:000293403000012}.

\hypertarget{ref-MacDonald2005}{}
MacDonald, Glen M., and Roslyn A. Case. 2005. ``Variations in the
Pacific Decadal Oscillation over the Past Millennium.''
\emph{Geophysical Research Letters} 32 (8): L08703.
\url{http://dx.doi.org/10.1029/2005GL022478}.

\hypertarget{ref-McAlpine2007}{}
McAlpine, C. A., J. Syktus, R. C. Deo, P. J. Lawrence, H. A. McGowan, I.
G. Watterson, and S. R. Phinn. 2007. ``Modeling the Impact of Historical
Land Cover Change on Australia's Regional Climate.'' \emph{Geophysical
Research Letters} 34 (22).
\href{\%3CGo\%20to\%20ISI\%3E://000251345100005}{\textless{}Go to ISI\textgreater{}://000251345100005}.

\hypertarget{ref-Mei2010}{}
Mei, R., and G. L. Wang. 2010. ``Rain Follows Logging in the Amazon?
Results from CAM3-CLM3.'' \emph{Climate Dynamics} 34 (7-8): 983--96.
\href{\%3CGo\%20to\%20ISI\%3E://WOS:000278088400005}{\textless{}Go to ISI\textgreater{}://WOS:000278088400005}.

\hypertarget{ref-Meneghini2007}{}
Meneghini, Belinda, Ian Simmonds, and Ian N. Smith. 2007. ``Association
Between Australian Rainfall and the Southern Annular Mode.''
\emph{International Journal of Climatology} 27 (1): 109--21.
\url{http://dx.doi.org/10.1002/joc.1370}.

\hypertarget{ref-Murphy2008}{}
Murphy, Bradley F., and Bertrand Timbal. 2008. ``A Review of Recent
Climate Variability and Climate Change in Southeastern Australia.''
\emph{International Journal of Climatology} 28 (7): 859--79.
\url{http://dx.doi.org/10.1002/joc.1627}.

\hypertarget{ref-Nair2011}{}
Nair, Udaysankar S., Y. Wu, J. Kala, T. J. Lyons, Sr. Pielke R. A., and
J. M. Hacker. 2011. ``The Role of Land Use Change on the Development and
Evolution of the West Coast Trough, Convective Clouds, and Precipitation
in Southwest Australia.'' \emph{Journal of Geophysical
Research-Atmospheres} 116 (D7). AGU: D07103.
\url{http://dx.doi.org/10.1029/2010JD014950}.

\hypertarget{ref-Narisma2007}{}
Narisma, G.T., J.A. Foley, R. Licker, and N. Ramankutty. 2007. ``Abrupt
Changes in Rainfall During the Twentieth Century.'' \emph{Geophysical
Research Letters} 34 (6): L06710.

\hypertarget{ref-Negri2004}{}
Negri, A. J., R. F. Adler, L. M. Xu, and J. Surratt. 2004. ``The Impact
of Amazonian Deforestation on Dry Season Rainfall.'' \emph{Journal of
Climate} 17 (6): 1306--19.
\href{\%3CGo\%20to\%20ISI\%3E://WOS:000220450100013\%20http://journals.ametsoc.org/doi/pdf/10.1175/1520-0442\%282004\%29017\%3C1306\%3ATIOADO\%3E2.0.CO\%3B2}{\textless{}Go to ISI\textgreater{}://WOS:000220450100013 http://journals.ametsoc.org/doi/pdf/10.1175/1520-0442\%282004\%29017\%3C1306\%3ATIOADO\%3E2.0.CO\%3B2}.

\hypertarget{ref-Nicholls2006}{}
Nicholls, N. 2006. ``Detecting and Attributing Australian Climate
Change: A Review.'' \emph{Australian Meteorological Magazine} 55 (3).
Bureau of Meteorology.: 199--211.

\hypertarget{ref-Oleson2004}{}
Oleson, K. W., G. B. Bonan, S. Levis, and M. Vertenstein. 2004.
``Effects of Land Use Change on North American Climate: Impact of
Surface Datasets and Model Biogeophysics.'' \emph{Climate Dynamics} 23
(2): 117--32.

\hypertarget{ref-Onoz2003}{}
Onoz, B., and M. Bayazit. 2003. ``The Power of Statistical Tests for
Trend Detection.'' \emph{Turkish Journal of Engineering and
Environmental Sciences} 27 (4): 247--51.

\hypertarget{ref-Otterman1990}{}
Otterman, J., A. Manes, S. Rubin, P. Alpert, and D. Starr. 1990. ``An
Increase of Early Rains in Southern Israel Following Land-Use Change?''
\emph{Boundary-Layer Meteorology} 53 (4): 333--51.

\hypertarget{ref-Pinto2009}{}
Pinto, E., Y. Shin, S. A. Cowling, and C. D. Jones. 2009. ``Past,
Present and Future Vegetation-Cloud Feedbacks in the Amazon Basin.''
\emph{Climate Dynamics} 32 (6): 741--51.
\href{\%3CGo\%20to\%20ISI\%3E://WOS:000264118400001}{\textless{}Go to ISI\textgreater{}://WOS:000264118400001}.

\hypertarget{ref-Pitman2007}{}
Pitman, A. J., and P. P. Hesse. 2007. ``The Significance of Large-Scale
Land Cover Change on the Australian Palaeomonsoon.'' \emph{Quaternary
Science Reviews} 26 (1-2): 189--200.
\url{http://www.sciencedirect.com/science/article/B6VBC-4MC71SN-1/2/0e1d4dff6e15a1de79cefbfa612f9b37}.

\hypertarget{ref-pitman_scale_2016}{}
Pitman, A. J., and R. Lorenz. 2016. ``Scale Dependence of the Simulated
Impact of Amazonian Deforestation on Regional Climate.''
\emph{Environmental Research Letters} 11 (9): 094025.
doi:\href{https://doi.org/10.1088/1748-9326/11/9/094025}{10.1088/1748-9326/11/9/094025}.

\hypertarget{ref-Pitman2004}{}
Pitman, A. J., G. T. Narisma, R. A. Pielke, and N. J. Holbrook. 2004.
``Impact of Land Cover Change on the Climate of Southwest Western
Australia.'' \emph{Journal of Geophysical Research-Atmospheres} 109
(D18).
\href{\%3CGo\%20to\%20ISI\%3E://WOS:000224126500001}{\textless{}Go to ISI\textgreater{}://WOS:000224126500001}.

\hypertarget{ref-Rstats2018}{}
R Core Team, 2018. \emph{R: A Language and Environment for Statistical
Computing}. Vienna, Austria: R Foundation for Statistical Computing.
\url{http://www.R-project.org/}.

\hypertarget{ref-Risbey2009}{}
Risbey, James S., Michael J. Pook, Peter C. McIntosh, Matthew C.
Wheeler, and Harry H. Hendon. 2009. ``On the Remote Drivers of Rainfall
Variability in Australia.'' \emph{Monthly Weather Review} 137 (10):
3233--53. \url{http://dx.doi.org/10.1175/2009MWR2861.1}.

\hypertarget{ref-Roy2008}{}
Roy, David P, Luigi Boschetti, Christopher O Justice, and J Ju. 2008.
``The Collection 5 Modis Burned Area Product---Global Evaluation by
Comparison with the Modis Active Fire Product.'' \emph{Remote Sensing of
Environment} 112 (9). Elsevier: 3690--3707.

\hypertarget{ref-Roy2005}{}
Roy, DP, Y Jin, PE Lewis, and CO Justice. 2005. ``Prototyping a Global
Algorithm for Systematic Fire-Affected Area Mapping Using Modis Time
Series Data.'' \emph{Remote Sensing of Environment} 97 (2). Elsevier:
137--62.

\hypertarget{ref-Roy2002}{}
Roy, DP, PE Lewis, and CO Justice. 2002. ``Burned Area Mapping Using
Multi-Temporal Moderate Spatial Resolution Data---A Bi-Directional
Reflectance Model-Based Expectation Approach.'' \emph{Remote Sensing of
Environment} 83 (1). Elsevier: 263--86.

\hypertarget{ref-saha_investigating_2016}{}
Saha, Subodh K., Paul A. Dirmeyer, and Thomas N. Chase. 2016.
``Investigating the Impact of Land-Use Land-Cover Change on Indian
Summer Monsoon Daily Rainfall and Temperature During 1951-2005 Using a
Regional Climate Model.'' \emph{Hydrology and Earth System Sciences} 20
(5): 1765.
\url{http://search.proquest.com/openview/c4242c9ecce075e3545390b1d00f2751/1?pq-origsite=gscholar\&cbl=105724}.

\hypertarget{ref-Saji1999}{}
Saji, NH, B.N. Goswami, PN Vinayachandran, and T. Yamagata. 1999. ``A
Dipole Mode in the Tropical Indian Ocean.'' \emph{Nature} 401 (6751):
360--63.

\hypertarget{ref-Sato2007}{}
Sato, T., F. Kimura, and A. S. Hasegawa. 2007. ``Vegetation and
Topographic Control of Cloud Activity over Arid/Semiarid Asia.''
\emph{Journal of Geophysical Research-Atmospheres} 112 (D24).
\href{\%3CGo\%20to\%20ISI\%3E://WOS:000252013700002\%20http://www.agu.org/journals/jd/jd0724/2006JD008129/2006JD008129.pdf}{\textless{}Go to ISI\textgreater{}://WOS:000252013700002 http://www.agu.org/journals/jd/jd0724/2006JD008129/2006JD008129.pdf}.

\hypertarget{ref-Schepen2012}{}
Schepen, Andrew, Q. J. Wang, and David Robertson. 2012. ``Evidence for
Using Lagged Climate Indices to Forecast Australian Seasonal Rainfall.''
\emph{Journal of Climate} 25 (4). American Meteorological Society:
1230--46.

\hypertarget{ref-Semazzi2001}{}
Semazzi, F. H. M., and Y. Song. 2001. ``A GCM Study of Climate Change
Induced by Deforestation in Africa.'' \emph{Climate Research} 17 (2):
169--82.

\hypertarget{ref-Smith2012}{}
Smith, Ian N., and Bertrand Timbal. 2012. ``Links Between Tropical
Indices and Southern Australian Rainfall.'' \emph{International Journal
of Climatology} 32 (1): 33--40.

\hypertarget{ref-Speer2011}{}
Speer, Milton, Lance Leslie, and Alexandre Fierro. 2011. ``Australian
East Coast Rainfall Decline Related to Large Scale Climate Drivers.''
\emph{Climate Dynamics} 36 (7): 1419--29.
\url{http://dx.doi.org/10.1007/s00382-009-0726-1}.

\hypertarget{ref-Fire2011}{}
The State Government of Victoria. 2011. ``Bushfire History.''

\hypertarget{ref-Timbal2006}{}
Timbal, B., and J.M. Arblaster. 2006. ``Land Cover Change as an
Additional Forcing to Explain the Rainfall Decline in the South West of
Australia.'' \emph{Geophysical Research Letters} 33 (7): L07717.

\hypertarget{ref-Townshend2011}{}
Townshend, J.R.G., M. Carroll, C. Dimiceli, R. Sohlberg, M. Hansen, and
R. DeFries. 2011. ``Vegetation Continuous Fields MOD44B, 2001 Percent
Tree Cover, Collection 5.'' University of Maryland, College Park,
Maryland, 2001.

\hypertarget{ref-Trenberth1999}{}
Trenberth, K. E. 1999. ``Atmospheric Moisture Recycling: Role of
Advection and Local Evaporation.'' \emph{Journal of Climate} 12 (5):
1368--81.
\href{\%3CGo\%20to\%20ISI\%3E://000080145700002}{\textless{}Go to ISI\textgreater{}://000080145700002}.

\hypertarget{ref-Verdon2004}{}
Verdon, Danielle C., Adam M. Wyatt, Anthony S. Kiem, and Stewart W.
Franks. 2004. ``Multidecadal Variability of Rainfall and Streamflow:
Eastern Australia.'' \emph{Water Resources Research} 40 (10): W10201.
\url{http://dx.doi.org/10.1029/2004WR003234}.

\hypertarget{ref-Wang2009}{}
Wang, Jingfeng, Frederic Chagnon, Earle Williams, Alan Betts, Nilton
Renno, Luiz Machado, Gautam Bisht, Ryan Knox, and Rafael Bras. 2009.
``Impact of Deforestation in the Amazon Basin on Cloud Climatology.''
\emph{Proceedings of the National Academy of Sciences of the United
States of America} 106 (10): 3670.
\url{http://ezproxy.library.usyd.edu.au/login?url=http://search.proquest.com/docview/201417978?accountid=14757}.

\hypertarget{ref-Westra2013}{}
Westra, Seth, Lisa V. Alexander, and Francis W. Zwiers. 2013. ``Global
Increasing Trends in Annual Maximum Daily Precipitation.'' \emph{Journal
of Climate} 26 (11). American Meteorological Society: 3904--18.
\url{http://dx.doi.org/10.1175/JCLI-D-12-00502.1}.

\hypertarget{ref-Westra2010}{}
Westra, Seth, and Ashish Sharma. 2010. ``An Upper Limit to Seasonal
Rainfall Predictability?'' \emph{Journal of Climate} 23 (12). American
Meteorological Society: 3332--51.
\href{http://ezproxy.library.usyd.edu.au/login?url=http://search.proquest.com/docview/814127501?accountid=14757\%20http://DD8GH5YX7K.search.serialssolutions.com/?SS_Source=3\&genre=article\&sid=ProQ:\&atitle=An+Upper+Limit+to+Seasonal+Rainfall+Predictability\%3F\&title=Journal+of+Climate\&issn=0894-8755\&date=2010-06-01\&volume=23\&issue=12\&spage=3332\&SS_docid=814127501\&author=Westra\%2C+Seth\%3BSharma\%2C+Ashish}{http://ezproxy.library.usyd.edu.au/login?url=http://search.proquest.com/docview/814127501?accountid=14757 http://DD8GH5YX7K.search.serialssolutions.com/?SS\_Source=3\&genre=article\&sid=ProQ:\&atitle=An+Upper+Limit+to+Seasonal+Rainfall+Predictability\%3F\&title=Journal+of+Climate\&issn=0894-8755\&date=2010-06-01\&volume=23\&issue=12\&spage=3332\&SS\_docid=814127501\&author=Westra\%2C+Seth\%3BSharma\%2C+Ashish}.

\hypertarget{ref-Whitley2002}{}
Whitley, Elise, and Jonathan Ball. 2002. ``Statistics Review 6:
Nonparametric Methods.'' \emph{Critical Care, London} 6 (6): 509--13.

\hypertarget{ref-Wilks2006}{}
Wilks, D. S. 2006. ``On `Field Significance' and the False Discovery
Rate.'' \emph{Journal of Applied Meteorology and Climatology} 45 (9).
American Meteorological Society: 1181--9.
\href{http://ezproxy.library.usyd.edu.au/login?url=http://search.proquest.com/docview/224582143?accountid=14757\%20http://DD8GH5YX7K.search.serialssolutions.com/?ctx_ver=Z39.88-2004\&ctx_enc=info:ofi/enc:UTF-8\&rfr_id=info:sid/ProQ\%3Amilitary\&rft_val_fmt=info:ofi/fmt:kev:mtx:journal\&rft.genre=article\&rft.jtitle=Journal+of+Applied+Meteorology+and+Climatology\&rft.atitle=On+\%22Field+Significance\%22+and+the+False+Discovery+Rate\&rft.au=Wilks\%2C+D+S\&rft.aulast=Wilks\&rft.aufirst=D\&rft.date=2006-09-01\&rft.volume=45\&rft.issue=9\&rft.spage=1181\&rft.isbn=\&rft.btitle=\&rft.title=Journal+of+Applied+Meteorology+and+Climatology\&rft.issn=15588424}{http://ezproxy.library.usyd.edu.au/login?url=http://search.proquest.com/docview/224582143?accountid=14757 http://DD8GH5YX7K.search.serialssolutions.com/?ctx\_ver=Z39.88-2004\&ctx\_enc=info:ofi/enc:UTF-8\&rfr\_id=info:sid/ProQ\%3Amilitary\&rft\_val\_fmt=info:ofi/fmt:kev:mtx:journal\&rft.genre=article\&rft.jtitle=Journal+of+Applied+Meteorology+and+Climatology\&rft.atitle=On+\%22Field+Significance\%22+and+the+False+Discovery+Rate\&rft.au=Wilks\%2C+D+S\&rft.aulast=Wilks\&rft.aufirst=D\&rft.date=2006-09-01\&rft.volume=45\&rft.issue=9\&rft.spage=1181\&rft.isbn=\&rft.btitle=\&rft.title=Journal+of+Applied+Meteorology+and+Climatology\&rft.issn=15588424}.

\hypertarget{ref-Wilks1997}{}
Wilks, DS. 1997. ``Resampling Hypothesis Tests for Autocorrelated
Fields.'' \emph{Journal of Climate} 10 (1): 65--82.

\hypertarget{ref-Wood2011}{}
Wood, S. N. 2011. ``Fast Stable Restricted Maximum Likelihood and
Marginal Likelihood Estimation of Semiparametric Generalized Linear
Models.'' \emph{Journal of the Royal Statistical Society (B)} 73 (1):
3--36.

\hypertarget{ref-Zanchettin2008}{}
Zanchettin, Davide, Stewart W. Franks, Pietro Traverso, and Mario
Tomasino. 2008. ``On Enso Impacts on European Wintertime Rainfalls and
Their Modulation by the Nao and the Pacific Multi-Decadal Variability
Described Through the Pdo Index.'' \emph{International Journal of
Climatology} 28 (8). John Wiley \& Sons, Ltd.: 995--1006.
\url{http://dx.doi.org/10.1002/joc.1601}.

\hypertarget{ref-Zeng2012}{}
Zeng, X. M., Z. H. Wu, S. Song, S. Y. Xiong, Y. Q. Zheng, Z. G. Zhou,
and H. Q. Liu. 2012. ``Effects of Land Surface Schemes on the Simulation
of a Heavy Rainfall Event by WRF.'' \emph{Chinese Journal of
Geophysics-Chinese Edition} 55 (1): 16--28.
\href{\%3CGo\%20to\%20ISI\%3E://WOS:000300129300002}{\textless{}Go to ISI\textgreater{}://WOS:000300129300002}.

\end{document}


